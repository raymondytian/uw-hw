%%%%%%%%%%%%%%%%%%%%%%%%%%%%% Define Article %%%%%%%%%%%%%%%%%%%%%%%%%%%%%%%%%%
\documentclass{report}
%%%%%%%%%%%%%%%%%%%%%%%%%%%%%%%%%%%%%%%%%%%%%%%%%%%%%%%%%%%%%%%%%%%%%%%%%%%%%%%

%%%%%%%%%%%%%%%%%%%%%%%%%%%%% Using Packages %%%%%%%%%%%%%%%%%%%%%%%%%%%%%%%%%%
\usepackage[margin=.75in,top=1in,bottom=.5in]{geometry}
\usepackage{mathtools}
\usepackage{geometry}
\usepackage{graphicx} 
\usepackage{amssymb}
\usepackage{amsmath}
\usepackage{amsthm}
\usepackage{empheq}
\usepackage{mdframed}
\usepackage{booktabs}
\usepackage{lipsum}
\usepackage{graphicx}
\usepackage{color}
\usepackage{psfrag}
\usepackage{pgfplots}
\usepackage{bm}
\usepackage{fancyhdr}
\usepackage{xpatch}
\usepackage{fix-cm}
\usepackage{titlesec}
\usepackage[english]{babel}
\usepackage[utf8x]{inputenc}
\usepackage{amsmath}
\usepackage{graphicx}
\usepackage{float}
\usepackage[colorinlistoftodos]{todonotes}
\usepackage{parskip}
\usepackage{tabularx}
\usepackage{adjustbox}
\usepackage{titlesec}
\usepackage{soul}
%%%%%%%%%%%%%%%%%%%%%%%%%%%%%%%%%%%%%%%%%%%%%%%%%%%%%%%%%%%%%%%%%%%%%%%%%%%%%%%

% Other Settings

%%%%%%%%%%%%%%%%%%%%%%%%%%%%%%%%%% Commands %%%%%%%%%%%%%%%%%%%%%%%%%%%%%%%%%%%
\newcommand{\newpg}{\vspace{5mm}}
\renewcommand{\chaptermark}[1]{\markboth{\MakeUppercase{#1}}{}}
%%%%%%%%%%%%%%%%%%%%%%%%%%%%%%%%%%%%%%%%%%%%%%%%%%%%%%%%%%%%%%%%%%%%%%%%%%%%%%%

%%%%%%%%%%%%%%%%%%%%%%%%%% Page Setting %%%%%%%%%%%%%%%%%%%%%%%%%%%%%%%%%%%%%%%
\geometry{a4paper}

%%%%%%%%%%%%%%%%%%%%%%%%%% Define some useful colors %%%%%%%%%%%%%%%%%%%%%%%%%%
\definecolor{ocre}{RGB}{243,102,25}
\definecolor{mygray}{RGB}{243,243,244}
\definecolor{deepGreen}{RGB}{26,111,0}
\definecolor{shallowGreen}{RGB}{235,255,255}
\definecolor{deepBlue}{RGB}{61,124,222}
\definecolor{shallowBlue}{RGB}{235,249,255}
\definecolor{gray75}{gray}{0.75}
%%%%%%%%%%%%%%%%%%%%%%%%%%%%%%%%%%%%%%%%%%%%%%%%%%%%%%%%%%%%%%%%%%%%%%%%%%%%%%%

%%%%%%%%%%%%%%%%%%%%%%%%%% Define an orangebox command %%%%%%%%%%%%%%%%%%%%%%%%
\newcommand\orangebox[1]{\fcolorbox{ocre}{mygray}{\hspace{1em}#1\hspace{1em}}}
%%%%%%%%%%%%%%%%%%%%%%%%%%%%%%%%%%%%%%%%%%%%%%%%%%%%%%%%%%%%%%%%%%%%%%%%%%%%%%%

%%%%%%%%%%%%%%%%%%%%%%%%%%%% English Environments %%%%%%%%%%%%%%%%%%%%%%%%%%%%%
\newtheoremstyle{mytheoremstyle}{3pt}{3pt}{\normalfont}{0cm}{\rmfamily\bfseries}{}{1em}{{\color{black}\thmname{#1}~\thmnumber{#2}}\thmnote{\,--\,#3}}
\newtheoremstyle{myproblemstyle}{3pt}{3pt}{\normalfont}{0cm}{\rmfamily\bfseries}{}{1em}{{\color{black}\thmname{#1}~\thmnumber{#2}}\thmnote{\,--\,#3}}
\theoremstyle{mytheoremstyle}
\newmdtheoremenv[linewidth=1pt,backgroundcolor=shallowGreen,linecolor=deepGreen,leftmargin=0pt,innerleftmargin=20pt,innerrightmargin=20pt,]{theorem}{Theorem}[section]
\theoremstyle{mytheoremstyle}
\newmdtheoremenv[linewidth=1pt,backgroundcolor=shallowBlue,linecolor=deepBlue,leftmargin=0pt,innerleftmargin=20pt,innerrightmargin=20pt,]{definition}{Definition}[section]
\theoremstyle{myproblemstyle}
\newmdtheoremenv[linecolor=black,leftmargin=0pt,innerleftmargin=10pt,innerrightmargin=10pt,]{problem}{Problem}[section]
%%%%%%%%%%%%%%%%%%%%%%%%%%%%%%%%%%%%%%%%%%%%%%%%%%%%%%%%%%%%%%%%%%%%%%%%%%%%%%%

%%%%%%%%%%%%%%%%%%%%%%%%%%%%%%% Plotting Settings %%%%%%%%%%%%%%%%%%%%%%%%%%%%%
\usepgfplotslibrary{colorbrewer}
\pgfplotsset{width=8cm,compat=1.9}
%%%%%%%%%%%%%%%%%%%%%%%%%%%%%%%%%%%%%%%%%%%%%%%%%%%%%%%%%%%%%%%%%%%%%%%%%%%%%%%

%%%%%%%%%%%%%%%%%%%%%%%%%%%%%%%%% Page Setup %%%%%%%%%%%%%%%%%%%%%%%%%%%%%%%%%%
\pagestyle{fancy}
\fancyhf{}
\fancyhead[L]{\nouppercase{\leftmark}}
\fancyhead[R]{\textsc{Raymond Tian}}
\fancyfoot[C]{\thepage}
\setlength{\footskip}{12pt}
\titleformat{\chapter}[display]{\filleft\Huge\bfseries}{\fontsize{100}{100}\selectfont\textcolor{black}\thechapter}{1ex}{}[]
\titleformat*{\section}{\LARGE\bfseries}
\DeclarePairedDelimiterX\set[1]\lbrace\rbrace{\def\given{\;\delimsize\vert\;}#1}
%%%%%%%%%%%%%%%%%%%%%%%%%%%%%%%%%%%%%%%%%%%%%%%%%%%%%%%%%%%%%%%%%%%%%%%%%%%%%%%

%%%%%%%%%%%%%%%%%%%%%%%%%%%%%%% Title & Author %%%%%%%%%%%%%%%%%%%%%%%%%%%%%%%%
\title{Notes on Introductory Discrete Mathematics}
\author{Raymond Tian}
\date{December 23, 2022}
%%%%%%%%%%%%%%%%%%%%%%%%%%%%%%%%%%%%%%%%%%%%%%%%%%%%%%%%%%%%%%%%%%%%%%%%%%%%%%%

\begin{document}
\begin{titlepage}
\newcommand{\HRule}{\rule{\linewidth}{0.5mm}}  
\begin{center}
\textsc{\LARGE University of Wisconsin---Madison}\\[1.5cm] 
\includegraphics[scale=.1]{../../images/uw-logo.png}\\[1cm] 
\textsc{\Large Introduction to Discrete Mathematics}\\[0.5cm] 
\textsc{\large Math 240}\\[0.5cm] 

\HRule \\[0.4cm]
{ \huge \bfseries Assignment 3}\\[0.4cm] 
\HRule \\[1.5cm]

\begin{minipage}{0.4\textwidth}
\begin{flushleft} \large
\emph{Author:}\\
Raymond \textsc{Tian}\\ 
\end{flushleft}

\end{minipage}\\[2cm]

{\large \today}\\[2cm] 

\vfill 

\end{center}
\end{titlepage}

\pagestyle{empty}
\newpage
\pagestyle{fancy}
%%%%%%%%%%%%%%%%%%%%%%%%%%%%%%%%%%%%%%%%%%%%%%%%%%%%%%%%%%%%%%%%%%%%%%%%%%%%%%%
\newpage
\section*{\Large{Question 1: Set identities [5 points]}}
Consider the sets $A = \{1, 2\}$ and $B = \{3, 4\}$. Verify that, for these sets, the following identies hold: 
\begin{enumerate}
    \item $A \setminus (B \cap A) = A$ and 
    \item $A \setminus (B \setminus A) = A$.
\end{enumerate} 
Prove that, for \textit{arbitary} sets $A$ and $B$, item 2 above remains true but item 1 may fail to hold.
\\[\baselineskip]
\textbf{Item 1:}
\begin{proof}
Let $A = \{1, 2\}$ and $B = \{3, 4\}$. Then $B \cap A = \emptyset$. Therefore $A \setminus (B \cap A) = A \setminus \emptyset = A$. Now instead if $A$ and $B$ are arbitary sets, then $A \setminus (B \cap A)$ is not necessarily $A$ because if $B \cap A \neq \emptyset$, then $\exists x \in A \: ( x \in B \cap A )$. Therefore the cardinality of $A \setminus (B \cap A)$ is strictly less than the cardinality of $A$, so $A \setminus (B \cap A) \neq A$ if $A \cap B \neq \emptyset$.
\end{proof}
\textbf{Item 2:}
\begin{proof}
Let $A = \{ 1, 2\}$ and $B = \{3, 4\}$. Then $B \setminus A = B$. Therefore $A \setminus (B \setminus A) = A \setminus B = A$. Now let us consider this identity if $A$ and $B$ are arbitary sets. For all $x$ in $A$, $x$ cannot be in $B \setminus A$. Since $x\in A$ and $x \notin B \ A$, then $x \in A \setminus (B \setminus A)$. Therefore $A \subseteq A \setminus (B \setminus A)$. By definition, $A \setminus (B \setminus A) \subseteq A$. Therefore $A \setminus (B \setminus A) = A$ because they are subspaces of each other.\end{proof} 
%%%%%%%%%%%%%%%%%%%%%%%%%%%%%%%%%%%%%%%%%%%%%%%%%%%%%%%%%%%%%%%%%%%%%%%%%%%%%%%
\newpage
\section*{\Large{Question 2: Injectivity, surjectivity, and bijectivity [5 points]}}
For each of the functions below, determine and prove whether or not it is injective, surjective, and bijective.
\begin{enumerate}
    \item $f:\{0,1\}^3 \rightarrow \{0, 1\}^4$ is given by adding a copy of the first bit to the end of the binary string. In other words $f(xyz) = xyzx$.
    \item Let $S = \{1, 2, 3\}$ and consider $g: \mathcal{P}(S) \rightarrow \{0, 1, 2, 3\}$ given by $g(A) = |A|$, where recall that for any set $A$, $|A|$ denotes its cardinality.
\end{enumerate}
\textbf{Item 1:}
\begin{proof}
    $f$ is injective because for any binary string $abc \in \{0, 1\}^3$, $f(abc) = abcx$ where some $x = a$. Therefore for any other binary string in $\{0, 1\}^3$ that is not $abc$, the output will not contain $abc$, so $f$ is injective because $\forall (x,y) \in \{0,1\}^3$ such that $x \neq y \: (f(x) \neq f(y))$. We will prove that $f$ is not surjective by contradiction. For the sake of contradiction, suppose $abc$ is a binary string in $\{0, 1\}^3$ such that $f(abc) = 0011$. $0011$ is in $\{0, 1\}^4$. Then $a$ must be equal to $0$ and $1$, which is impossible. Therefore $f$ is not surjective. Since $f$ is not injective and surjective, then $f$ is not bijective.
\end{proof} 
\textbf{Item 2:}
\begin{proof}
    We will prove that $g$ is not injective by contradiction. For the sake of contradiciton, suppose that $g$ is injective. Then $\forall (A_1, A_2) \in \mathcal{P}(S)$ such that $S = \{1, 2, 3\}$ and $A_1 \neq A_2 \: (g(A_1) \neq g(A_2))$. However, suppose that $A_1 = \{1\}$ and $A_2 = \{2\}$. Then $g(A_1) = |A_1| = 1$ and $g(A_2) = |A_2| = 1$. Therefore $g(A_1) = g(A_2)$, which is a contradiction, so $g$ is not injective. We will prove that $g$ is surjective by exhaustion. The codomain of $g$ is $0, 1, 2,$ and $3$. For $g$ to be surjective, $\forall x \in \{0, 1, 2, 3\} \: \exists A \in \mathcal{P}(S) \: (g(A)=x)$. Let $A = \emptyset$, then $g(A) = |\emptyset| = 0$. Let $A = \{1\}$, then $g(A) = |{1}| = 1$. Let $A = \{1, 2\}$, then $g(A) = |\{1, 2\}| = 2.$ Finally, let $A = \{1, 2, 3\}$, then $g(A) = |\{1, 2, 3\}| = 3.$ All of these values of $A$ are in $\mathcal{P}(S)$ and all values in $\{0, 1, 2, 3\}$ is equal to $g(A)$ for some A. Therefore $g$ is surjective. Since $g$ is not injective and surjective, $g$ is not bijective.  
\end{proof}
%%%%%%%%%%%%%%%%%%%%%%%%%%%%%%%%%%%%%%%%%%%%%%%%%%%%%%%%%%%%%%%%%%%%%%%%%%%%%%%
\end{document}
