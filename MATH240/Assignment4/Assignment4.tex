%%%%%%%%%%%%%%%%%%%%%%%%%%%%% Define Article %%%%%%%%%%%%%%%%%%%%%%%%%%%%%%%%%%
\documentclass{report}
%%%%%%%%%%%%%%%%%%%%%%%%%%%%%%%%%%%%%%%%%%%%%%%%%%%%%%%%%%%%%%%%%%%%%%%%%%%%%%%

%%%%%%%%%%%%%%%%%%%%%%%%%%%%% Using Packages %%%%%%%%%%%%%%%%%%%%%%%%%%%%%%%%%%
\usepackage[margin=.75in,top=1in,bottom=.5in]{geometry}
\usepackage{mathtools}
\usepackage{geometry}
\usepackage{graphicx} 
\usepackage{amssymb}
\usepackage{amsmath}
\usepackage{amsthm}
\usepackage{empheq}
\usepackage{mdframed}
\usepackage{booktabs}
\usepackage{lipsum}
\usepackage{graphicx}
\usepackage{color}
\usepackage{psfrag}
\usepackage{pgfplots}
\usepackage{bm}
\usepackage{fancyhdr}
\usepackage{xpatch}
\usepackage{fix-cm}
\usepackage{titlesec}
\usepackage[english]{babel}
\usepackage[utf8x]{inputenc}
\usepackage{amsmath}
\usepackage{graphicx}
\usepackage{float}
\usepackage[colorinlistoftodos]{todonotes}
\usepackage{parskip}
\usepackage{tabularx}
\usepackage{adjustbox}
\usepackage{titlesec}
\usepackage{soul}
%%%%%%%%%%%%%%%%%%%%%%%%%%%%%%%%%%%%%%%%%%%%%%%%%%%%%%%%%%%%%%%%%%%%%%%%%%%%%%%

% Other Settings

%%%%%%%%%%%%%%%%%%%%%%%%%%%%%%%%%% Commands %%%%%%%%%%%%%%%%%%%%%%%%%%%%%%%%%%%
\newcommand{\newpg}{\vspace{5mm}}
\renewcommand{\chaptermark}[1]{\markboth{\MakeUppercase{#1}}{}}
%%%%%%%%%%%%%%%%%%%%%%%%%%%%%%%%%%%%%%%%%%%%%%%%%%%%%%%%%%%%%%%%%%%%%%%%%%%%%%%

%%%%%%%%%%%%%%%%%%%%%%%%%% Page Setting %%%%%%%%%%%%%%%%%%%%%%%%%%%%%%%%%%%%%%%
\geometry{a4paper}

%%%%%%%%%%%%%%%%%%%%%%%%%% Define some useful colors %%%%%%%%%%%%%%%%%%%%%%%%%%
\definecolor{ocre}{RGB}{243,102,25}
\definecolor{mygray}{RGB}{243,243,244}
\definecolor{deepGreen}{RGB}{26,111,0}
\definecolor{shallowGreen}{RGB}{235,255,255}
\definecolor{deepBlue}{RGB}{61,124,222}
\definecolor{shallowBlue}{RGB}{235,249,255}
\definecolor{gray75}{gray}{0.75}
%%%%%%%%%%%%%%%%%%%%%%%%%%%%%%%%%%%%%%%%%%%%%%%%%%%%%%%%%%%%%%%%%%%%%%%%%%%%%%%

%%%%%%%%%%%%%%%%%%%%%%%%%% Define an orangebox command %%%%%%%%%%%%%%%%%%%%%%%%
\newcommand\orangebox[1]{\fcolorbox{ocre}{mygray}{\hspace{1em}#1\hspace{1em}}}
%%%%%%%%%%%%%%%%%%%%%%%%%%%%%%%%%%%%%%%%%%%%%%%%%%%%%%%%%%%%%%%%%%%%%%%%%%%%%%%

%%%%%%%%%%%%%%%%%%%%%%%%%%%% English Environments %%%%%%%%%%%%%%%%%%%%%%%%%%%%%
\newtheoremstyle{mytheoremstyle}{3pt}{3pt}{\normalfont}{0cm}{\rmfamily\bfseries}{}{1em}{{\color{black}\thmname{#1}~\thmnumber{#2}}\thmnote{\,--\,#3}}
\newtheoremstyle{myproblemstyle}{3pt}{3pt}{\normalfont}{0cm}{\rmfamily\bfseries}{}{1em}{{\color{black}\thmname{#1}~\thmnumber{#2}}\thmnote{\,--\,#3}}
\theoremstyle{mytheoremstyle}
\newmdtheoremenv[linewidth=1pt,backgroundcolor=shallowGreen,linecolor=deepGreen,leftmargin=0pt,innerleftmargin=20pt,innerrightmargin=20pt,]{theorem}{Theorem}[section]
\theoremstyle{mytheoremstyle}
\newmdtheoremenv[linewidth=1pt,backgroundcolor=shallowBlue,linecolor=deepBlue,leftmargin=0pt,innerleftmargin=20pt,innerrightmargin=20pt,]{definition}{Definition}[section]
\theoremstyle{myproblemstyle}
\newmdtheoremenv[linecolor=black,leftmargin=0pt,innerleftmargin=10pt,innerrightmargin=10pt,]{problem}{Problem}[section]
%%%%%%%%%%%%%%%%%%%%%%%%%%%%%%%%%%%%%%%%%%%%%%%%%%%%%%%%%%%%%%%%%%%%%%%%%%%%%%%

%%%%%%%%%%%%%%%%%%%%%%%%%%%%%%% Plotting Settings %%%%%%%%%%%%%%%%%%%%%%%%%%%%%
\usepgfplotslibrary{colorbrewer}
\pgfplotsset{width=8cm,compat=1.9}
%%%%%%%%%%%%%%%%%%%%%%%%%%%%%%%%%%%%%%%%%%%%%%%%%%%%%%%%%%%%%%%%%%%%%%%%%%%%%%%

%%%%%%%%%%%%%%%%%%%%%%%%%%%%%%%%% Page Setup %%%%%%%%%%%%%%%%%%%%%%%%%%%%%%%%%%
\pagestyle{fancy}
\fancyhf{}
\fancyhead[L]{\nouppercase{\leftmark}}
\fancyhead[R]{\textsc{Raymond Tian}}
\fancyfoot[C]{\thepage}
\setlength{\footskip}{12pt}
\titleformat{\chapter}[display]{\filleft\Huge\bfseries}{\fontsize{100}{100}\selectfont\textcolor{black}\thechapter}{1ex}{}[]
\titleformat*{\section}{\LARGE\bfseries}
\DeclarePairedDelimiterX\set[1]\lbrace\rbrace{\def\given{\;\delimsize\vert\;}#1}
%%%%%%%%%%%%%%%%%%%%%%%%%%%%%%%%%%%%%%%%%%%%%%%%%%%%%%%%%%%%%%%%%%%%%%%%%%%%%%%

%%%%%%%%%%%%%%%%%%%%%%%%%%%%%%% Title & Author %%%%%%%%%%%%%%%%%%%%%%%%%%%%%%%%
\title{Notes on Introductory Discrete Mathematics}
\author{Raymond Tian}
\date{December 23, 2022}
%%%%%%%%%%%%%%%%%%%%%%%%%%%%%%%%%%%%%%%%%%%%%%%%%%%%%%%%%%%%%%%%%%%%%%%%%%%%%%%

\begin{document}
\begin{titlepage}
\newcommand{\HRule}{\rule{\linewidth}{0.5mm}}  
\begin{center}
\textsc{\LARGE University of Wisconsin---Madison}\\[1.5cm] 
\includegraphics[scale=.1]{../../images/uw-logo.png}\\[1cm] 
\textsc{\Large Introduction to Discrete Mathematics}\\[0.5cm] 
\textsc{\large Math 240}\\[0.5cm] 

\HRule \\[0.4cm]
{ \huge \bfseries Assignment 4}\\[0.4cm] 
\HRule \\[1.5cm]

\begin{minipage}{0.4\textwidth}
\begin{flushleft} \large
\emph{Author:}\\
Raymond \textsc{Tian}\\ 
\end{flushleft}

\end{minipage}\\[2cm]

{\large \today}\\[2cm] 

\vfill 

\end{center}
\end{titlepage}

\pagestyle{empty}
\newpage
\pagestyle{fancy}
%%%%%%%%%%%%%%%%%%%%%%%%%%%%%%%%%%%%%%%%%%%%%%%%%%%%%%%%%%%%%%%%%%%%%%%%%%%%%%%
\newpage
\section*{Question 1: Examples of big O notation [5 points]}
Prove the folowing statements.
\begin{enumerate}
    \item $n^3=O(e^n)$
        \begin{proof}
            Suppose that $n_0 = 1$ and $c = 24$. We will show that for any $n \geq 1$, $n^3 \leq 24\cdot e^n$. For every integer value of $n$ such that $n \geq 1$, $n^3 \leq n^4.$ The Taylor expansion of the exponential function is given by $$e^n=\sum_{i=0}^{\infty} \frac{n^k}{k!}=1+n+\frac{n^2}{2}+\frac{n^3}{6}+\frac{n^4}{24}+\ldots$$ Since $n \geq 1$, then $n^4 < 24\cdot e^n$. Combining these two inequalities gives us $$n^3 \leq n^4 < 24\cdot e^n$$ Therefore for $n \geq 1$, $n^3 \leq 24\cdot e^n$, so $n^3 = O(e^n)$.
        \end{proof}
    \item $n^3+3n^2-4=\Theta(n^3)$
        \begin{proof}
            Suppose that $n_0 = 2$ and $c = 1$. We will show that for any $n \geq 2$, $n^3+3n^2-4 \geq n^3$. Since $n \geq 2$, then $n^3 + 3n^2 - 4 \geq n^3 + 8 > n^3$. Therefore for $n \geq 2$, $n^3+3n^2-4 \geq n^3$, so $n^3+3n^2-4=\Omega (n^3)$. Suppose that $n_0 = 1$ and $c = 100$. We will show that for any $n \geq 1$, $n^3+3n^2-4 \leq 100n^3$. Since $n \geq 1$, then $n^3+3n^2-4 \leq 100n^3$. Therefore for $n \geq 1$, $n^3 + 3n^2 - 4 \leq c\cdot n^3$, so $n^3+3n^2-4=O(n^3)$. Since $n^3+3n^2-4=O(n^3)$ and $n^3+3n^2-4=\Omega(n^3)$, then by definition, $n^3+3n^2-4=\Theta(n^3)$.
        \end{proof}
    \item $n^{3/2} \neq \Omega(n^2)$
        \begin{proof}
            Suppose that $n_0 = 1$ and $c = 1$. For the sake of contradiction, we assume that $n^{3/2} = \Omega(n^2)$. Then for every $n \geq 1$, $n^{3/2} \geq n^2$. However if we set $n = 2$, then the previous inequality becomes $2\sqrt{2} \geq 4$, which is false. Therefore by contradiction, $n^{3/2} \neq \Omega(n^2)$.
        \end{proof}
\end{enumerate}
%%%%%%%%%%%%%%%%%%%%%%%%%%%%%%%%%%%%%%%%%%%%%%%%%%%%%%%%%%%%%%%%%%%%%%%%%%%%%%%
\newpage
\section*{Question 2: Some rules of big O notation [5 points]}
\begin{enumerate}
    \item $f=O(g)$ if and only if $g=\Omega(f)$.
        \begin{proof}
            We will prove this statement by proving both sides of the biconditional. If $f = O(g)$, then there exists real numbers $n_0 > 0$ and $c > 0$, such that for every integer $n \geq n_0$, $f(n) \leq c\cdot g(n)$, which is equivalent to $c\cdot g(n) \geq f(n)$. Since $c$ is positive, dividing both sides of the previous inequality by $c$ gives us $g(n) \geq \frac{1}{c}\cdot f(n)$. Since $\frac{1}{c}$ is a real number, then $g = \Omega(f).$ Therefore $f = O(g) \implies g = \Omega(f)$. Now, if $g = \Omega(f)$, then there exists real numbers $n_0 > 0$ and $c > 0$, such that for every integer $n \geq n _0$, then $g(n) \geq c\cdot f(n)$, which is equivalent to $c\cdot f(n) \leq g(n)$. Since $c$ is positive, dividing both sides of the previous inequality by $c$ gives us $f(n) \leq \frac{1}{c}\cdot g(n)$. Since $\frac{1}{c}$ is a real number, then $f = O(g)$. Therefore $g=\Omega(f)\implies f=O(g)$. Since $f=O(g)\implies g=\Omega(f)$ and $g=\Omega(f)\implies f=O(g)$, then $f=O(g)\iff g=\Omega(f)$.
        \end{proof}
    \item If $f_1=\Omega(g_1)$ and $f_2=\Omega(g_2)$ then $f_1f_2=\Omega(g_1g_2)$.
        \begin{proof}
            If $f_1=\Omega(g_1)$, there exists real numbers $n_{1_0} > 0$ and $c_1 > 0$, such that for every integer $n_1 \geq n_{1_0}$, $f_1(n_1)\geq c_1\cdot g_1(n_1)$. Similarly, if $f_2=\Omega(g_2)$, there exists real numbers $n_{2_0} > 0$ and $c_2 > 0$, such that for every integer $n_2 \geq n_{2_0}$, $f_2(n_2)\geq c_2\cdot g_2(n_2)$. Multiplying these two inequalities together gives us 
            \begin{align}
                f_1(n_1)f_2(n_2)&\geq (c_1\cdot g_1(n_1))\cdot(c_2\cdot g_2(n_2)) \\
                f_1(n_1)f_2(n_2)&\geq c_1c_2\cdot g_1(n_1)\cdot g_2(n_2)
            \end{align}
        Since $c1$ and $c2$ are both real numbers, then $c1c2$ is a real number. Additionally, we can assume without loss of generality that if $n_{1_0} \leq n_{2_0}$, then $$f_1(n)f_2(n) \geq c\cdot g_1(n)g_2(n)$$ where $c=c1c2$ and for any integer $n\geq n_{2_0}$. Therefore $f_1f_2 = \Omega(g_1g_2)$.
        \end{proof}
    \item If $f=O(g)$ and $g=O(h)$ then $f=O(h)$.
        \begin{proof}
            If $f=O(g)$ there exists positive real numbers $n_0,c$ such that for every integer $n\geq n_0, f(n)\leq c\cdot g(n)$. If $g=O(h)$ there exists positive real numbers $m_0,d$ such that for every integer $m\geq m_0, g(m)\leq d\cdot h(m)$. Dividing both sides of the first inequality by $c$ gives us $\frac{f(n)}{c}\leq g(n)$. Subbing this value of $g(n)$ into the second inequality gives us $\frac{f(m)}{c}\leq d\cdot h(m)$, which is equivallent to $f(m)\leq c\cdot d\cdot h(m)$. Since $c,d$ are integers, $c\cdot d$ is an integer, therefore $f=O(h)$.
        \end{proof}
\end{enumerate}
%%%%%%%%%%%%%%%%%%%%%%%%%%%%%%%%%%%%%%%%%%%%%%%%%%%%%%%%%%%%%%%%%%%%%%%%%%%%%%%
\newpage
\section*{Question 3: Examples of finite state machines [5 points]}
Note that I will only be writing the answers for Question 3.
\begin{enumerate}
    \item
        (a) $000\rightarrow 000$\\
        (b) $010\rightarrow 000$\\
        (c) $101\rightarrow 100$\\
        The output string has leading 1s that are in the same position as the input string's leading 1s. The remainder of the output string are 0s.
    \item
        (a) $100011\rightarrow\text{No}$\\
        (b) $1111\rightarrow\text{Yes}$\\
        (c) $0010\rightarrow\text{No}$\\
        (d) $1100\rightarrow\text{No}$\\
        The FSM only accepts input strings where every single character of the input string is 1.
    \item
        (a) $100011\rightarrow\text{No}$\\
        (b) $0000\rightarrow\text{No}$\\
        (c) $0010\rightarrow\text{No}$\\
        (d) $1100\rightarrow\text{Yes}$\\
        The FSM only accepts input strings where the first character is 1 and the last character is 0.
\end{enumerate}
%%%%%%%%%%%%%%%%%%%%%%%%%%%%%%%%%%%%%%%%%%%%%%%%%%%%%%%%%%%%%%%%%%%%%%%%%%%%%%%
\newpage
\section*{Question 4: Designing finite state machines [5 points]}
I will only be writing the answers. Since the problem did not specify to draw the FSMs, \textbf{I will write the proper notation for the FSMs in the form of tuples.}
\begin{enumerate}
    \item This FSM will be defined by the tuple $(Q, q_0, I, A, \delta)$ where $Q$ represents the finite set of states, $q_0\in Q$ represents the starting state, $I$ represents the finset set of inputs, $A\subseteq Q$ represents the accepting states, and $\delta\colon Q\times I\rightarrow Q$ represents the transition function.
        Let $Q=\{\alpha,\beta,\gamma\}$, $q_0=\alpha$, $I=\{0,1\}$ and $A=\alpha$. Let $\delta$ be defined by:
        \begin{align}
            \forall q\in Q\:(&\delta(q,0)=q) \\
            &\delta(\alpha,1)=\beta \\
            &\delta(\beta,1)=\gamma \\
            &\delta(\gamma,1)=\alpha
        \end{align} This is a valid FSM.
    \item This FSM will be defined by the tuple $(Q, q_0, I, O, \delta)$ where $Q$ represents the finite set of states, $q_0\in Q$ represents the starting state, $I$ represents the finite set of inputs, $O$ represents the finite set of outputs, and $\delta\colon Q\times I\rightarrow Q\times O$ represents the transition function.
        Let $Q=\{alpha\}$, $q_0=\alpha$, $I=\{0,1\}$, and $O=\{0,1\}$. Let $\delta$ be defined by:$$\forall i\in I\:(\delta(\alpha,i)=(\alpha,i))$$
        This is a valid FSM.
\end{enumerate}
%%%%%%%%%%%%%%%%%%%%%%%%%%%%%%%%%%%%%%%%%%%%%%%%%%%%%%%%%%%%%%%%%%%%%%%%%%%%%%%
\end{document}
